\usepackage{amsmath}
\usepackage{amssymb}
\usepackage{amsthm}
\usepackage{bm}
\usepackage{enumitem}
\usepackage{float}
\usepackage{graphicx}
\usepackage[procnames]{listings}
\usepackage{mathrsfs}  % \mathscr{[Capital Letter]}
\usepackage{mathtools}
\usepackage{parskip}
\usepackage{xparse}

\graphicspath{ {./img/} }

% Use with \eqref{...} to show equation numbers only for equations that are referenced
\mathtoolsset{showonlyrefs}

\newtheorem{theorem}{Theorem}[section]
\newtheorem{corollary}{Corollary}[theorem]
\newtheorem{lemma}{Lemma}

\newtheoremstyle{case}%
   {7pt}%
   {7pt}%
   {}%
   {}%
   {\bfseries}%
   {:}%
   {.5em}%
   {}

\newtheoremstyle{claim}%
   {7pt}%
   {7pt}%
   {}%
   {}%
   {\bfseries}%
   {.}%
   {.5em}%
   {}

\newtheoremstyle{remark}%
   {7pt}%
   {7pt}%
   {}%
   {}%
   {\itshape}%
   {.}%
   {.5em}%
   {}

\theoremstyle{case}
\newtheorem{case}{Case}

\theoremstyle{claim}
\newtheorem*{claim}{Claim}

\theoremstyle{remark}
\newtheorem*{remark}{Remark}

\makeatletter
\newcommand*{\declarecommand}{%
    \@star@or@long\declare@command
}
\newcommand*{\declare@command}[1]{%
    \provide@command{#1}{}%
    % \let#1\@empty % would be more efficient, but without error checking
    \RenewDocumentCommand{#1}%
}
\makeatother

% Global Custom Commands
\renewcommand{\a}{\alpha}
\newcommand{\Bythrm}[1]{By Theorem #1 of the text}
\renewcommand{\b}{\beta}
\newcommand{\bythrm}[1]{by Theorem #1 of the text}
\newcommand{\C}{\mathbb{C}}
\newcommand{\cC}{\mathcal{C}}
\newcommand{\cN}{\mathcal{N}}
\newcommand{\cR}{\mathcal{R}}
\newcommand{\cQ}{\mathcal{Q}}
\newcommand{\cZ}{\mathcal{Z}}
\renewcommand{\cal}[1]{\mathcal{#1}}
\newcommand{\com}[1]{\text{\scriptsize(#1)}}
\renewcommand{\d}{\delta}
\newcommand{\e}{\ensuremath{\epsilon}}
\declarecommand{\es}{}{\ensuremath{\emptyset}}
\newcommand{\h}[1]{\ensuremath{\hat{#1}}}
\declarecommand{\iff}{}{if and only if}
\renewcommand{\l}{\ell}
\declarecommand{\ker}{}{\ensuremath{\text{ker}}}
\newcommand{\la}{\leftarrow}
\newcommand{\N}{\mathbb{N}}
\declarecommand{\nb}{}{The solution to this problem is hand-written and can be found on the attached notebook paper.}
\declarecommand{\novs}{O{0.7cm}}{\vspace{-#1}}
\newcommand\numberthis{\addtocounter{equation}{1}\tag{\theequation}}
\declarecommand{\o}{}{\ensuremath{\omega}}
\newcommand{\R}{\mathbb{R}}
\newcommand{\ra}{\rightarrow}
\newcommand{\s}[1]{\ensuremath{\{#1\}}}
\newcommand{\set}[1]{\ensuremath{\{#1\}}}
\newcommand{\seq}[1]{\left\{#1\right\}_{n=1}^{\infty}}
\newcommand{\struct}[1]{\langle #1 \rangle}
\newcommand{\st}[1]{\ensuremath{\langle #1 \rangle}}
\newcommand{\sub}{\ensuremath{\subset}}
\newcommand{\sube}{\ensuremath{\subseteq}}
\renewcommand{\t}{\theta}
\declarecommand{\tx}{m}{\ensuremath{\text{#1}}}
\newcommand{\vphi}{\varphi}
\declarecommand{\w}{}{\ensuremath{\omega}}
\newcommand{\Q}{\mathbb{Q}}
\newcommand{\Z}{\mathbb{Z}}
\declarecommand{\=}{}{&=}
